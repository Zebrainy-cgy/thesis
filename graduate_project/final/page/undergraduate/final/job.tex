\cleardoublepage{}
\sectionnonum{本科生毕业论文(设计)任务书}

{
    \bfseries
    \noindent 一、题目:基于对流扩散方程关于物理信息机器学习算法的研究

    \noindent 二、指导教师对毕业论文(设计)的进度安排及任务要求:

    (1)2023.11.12秋第八周前:学生,导师双向选择,确定选题,导师下达任务书,对进度,文献和开题提出要求。

    (2)2023.11.19冬第一周前:学生确认任务书,对确定的课题搜集相关文献资料,了解问题的背景,应用,研究历史与现状。从中确定论文最终题目。

    (3)2024.01.07冬第八周前,对确定的题目进一步开展学习,包括所必需的基础知识以及近几年涉及此问题的文章。初步撰写并完成开题报告,文献综述,外文翻译。

    (4)2024.02.25春学期开学前,上传符合规范格式要求的三合一至教务系统。

    (5)2024.03.03春第一周前,开题答辩,并根据答辩小组以金进行修改,将定稿的开题报告,文献综述,外文翻译上传至教务系统。

    (6)2024.03.31春第五周前:做中期检查报告。

    (7)2024.05.12夏第三周前:完成论文初稿,进行论文稿的修改并最终完成,项导师提交论文终稿。
    
    (8)2024.05.19夏第四周前:导师评阅,学生提交导师填写评语和签字的”毕业论文考核表“以及符合规范格式要求的送审论文。

    (9)2024.05.26夏第五周前:毕业论文专家评阅。评阅结果有修改意见的更几乎评阅意见对论文进行修改。

    (10)2024.06.02夏第六周:组织毕业论文答辩。提交最终版毕业论文并将论文上传到教务系统。
    % \vskip 50mm

    \noindent 起讫日期 \quad 2023 年 \quad  11月 \quad1  日 \quad 至 \quad  \quad 2024 年 \quad 6 月  \quad 2日
    \begin{flushright}
        \bfseries \zihao{-4}
            指导教师(签名) \underline{\multido{}{5}{\quad}} 职称 \underline{\multido{}{5}{\quad}}
    \end{flushright}

    \noindent 三、系或研究所审核意见:

    % \mbox{} \vfill
    同意该计划!
    \signature{负责人(签名)}
}
